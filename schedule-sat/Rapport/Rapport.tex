\documentclass[a4paper,10pt]{article}
\usepackage[utf8]{inputenc}

\usepackage[a4paper]{geometry}
\usepackage{amsmath,amsthm,amssymb,latexsym,amsfonts}
\usepackage{hyperref}
\geometry{hscale=0.75,vscale=0.75,centering}
\usepackage{graphicx}
\usepackage{listings}

\usepackage{tikz}

\begin{document}

  \lstset{language=C,
  basicstyle=\ttfamily\footnotesize,        % the size of the fonts that are used for the code
  breakatwhitespace=false,         % sets if automatic breaks should only happen at whitespace
  breaklines=true,                 % sets automatic line breaking
  commentstyle=\color{cyan},    % comment style
  keepspaces=true,                 % keeps spaces in text, useful for keeping indentation of code (possibly needs columns=flexible)
  keywordstyle=\color{blue},       % keyword style
  numbers=left,                    % where to put the line-numbers; possible values are (none, left, right)
  numbersep=5pt,                   % how far the line-numbers are from the code
  numberstyle=\tiny\color{blue}, % the style that is used for the line-numbers
  rulecolor=\color{black},         % if not set, the frame-color may be changed on line-breaks within not-black text (e.g. comments (green here))
  %showspaces=false,                % show spaces everywhere adding particular underscores; it overrides 'showstringspaces'
  showstringspaces=false,          % underline spaces within strings only
  showtabs=false,                  % show tabs within strings adding particular underscores
  stepnumber=5,                    % the step between two line-numbers. If it's 1, each line will be numbered
  stringstyle=\color{red},     % string literal style
  tabsize=2,                       % sets default tabsize to 2 spaces
} 

\begin{titlepage}
    \begin{center}

	\begin{tikzpicture}[remember picture, overlay]
	  \node [anchor=north east, inner sep=0pt]  at (current page.north east)
	     {\includegraphics[height=3cm]{Banniere_ULB.png}};
	\end{tikzpicture}
	
        {\Large \textbf{\textsc{Faculté des Sciences}}\\
        \textbf{\textsc{Département d'Informatique}}}

        \vfill{}\vfill{}

        \begin{center}
            \Huge{INFO-H-302}
                \Huge{Projet: Logique propositionnelle et utilisation de l'outil MiniSAT}
        \end{center}
        \Huge{\par}
        \begin{center}
            \large{
                \textsc{Bertha} David \\
                \textsc{Tio Nogueras} Gérard \\
            }
        \end{center}
        \Huge{\par}

        \vfill{}\vfill{}
        \large{\par}

        \vfill{}\vfill{}\enlargethispage{3cm}

        \begin{figure} [h!]
             \centering
	    \includegraphics[width=4cm]{Sigle_ULB.png}
	\end{figure}

        \textbf{Année académique 2014~-~2015}
        
    \end{center}
\end{titlepage}
\newpage
\tableofcontents
\pagebreak
\section{Introduction}
\section{Questions}
\subsection{Définition des contraints de $\mu$}
\begin{enumerate}
\item
Chaque examen a bien lieu 1 fois.
\item
Chaque examen a lieu une seule fois pour une seule salle et une seule période.
\item
Les examens ne peuvent pas se dérouler dans des pièces de capacité insuffisante.
\item
Les examens d'un même étudiant ne peuvent pas avoir lieu en même temps.
\item
Les examens d'un même professeur ne peuvent pas avoir lieu en même temps.
\item
Deux examens différents ne peuvent pas avoir lieur dans la même salle en même temps.
\end{enumerate}
\subsection{Ecriture des contraintes sous forme logique(langage mathématique)}
\begin{enumerate}
\item
\fbox{\begin{minipage}{1.0\textwidth}
\begin{equation}
\forall x \in X,  \exists s \in S,  \exists t \in \{1,..,T\} | \mu(x) = ( s, t ). \\
\end{equation}
\end{minipage}}
\item
\fbox{\begin{minipage}{1.0\textwidth}
\begin{equation}
\forall x \in X,  \exists! s \in S,  \exists! t \in \{1,..,T\} | \mu(x) = ( s, t ). \\
\end{equation}
\end{minipage}}
\item
\fbox{\begin{minipage}{1.0\textwidth}
Création d'une fonction qui nous donne le nombre d'étudiants par examen.
\begin{equation}
l: X -> \mathbb{N} | l(x) = n \Leftrightarrow \lVert \{  e \in E | x \in a(e) \} \rVert  = n
\end{equation}
\begin{equation}
\forall x \in X, \forall s \in S | e(x) > c(s), \forall t \in \{1,..,T\} |  \mu(x) \neq  (s,t). \\
\end{equation}
\end{minipage}}
\item
\fbox{\begin{minipage}{1.0\textwidth}
\begin{equation}
\forall e \in E,\forall x1,x2 \in X | x1,x2 \in a(e), \forall t1,t2 \in \{1,..,T\}, \forall s1,s2 \in S 
\end{equation}
\begin{equation}
 | \mu(x1) = (s1,t1) \land \mu(x2) = (s2,t2) \rightarrow  t1 \neq t2.  \\
\end{equation}
\end{minipage}}
\item
\fbox{\begin{minipage}{1.0\textwidth}
\begin{equation}
\forall p \in P,\forall x1,x2 \in X | x1,x2 \in b(e), \forall t1,t2 \in \{1,..,T\}, \forall s1,s2 \in S, 
\end{equation}
\begin{equation}
 | \mu(x1) = (s1,t1) \land \mu(x2) = (s2,t2) \rightarrow  t1 \neq t2.  \\
\end{equation}
\end{minipage}}
\item
\fbox{\begin{minipage}{1.0\textwidth}
\begin{equation}
\forall x1, x2 \in X, \forall s \in S, \forall t \in \{1,..,T\} | ( \mu(x1) = (s,t) \land \mu(x2) = (s,t) ) \rightarrow x1 = x2
\end{equation}
\end{minipage}}
\end{enumerate}
\subsection{Construction de la formule en FNC}
Nous manipulerons des indices commençant à 0 et non à 1 comme dans l'input. Nous faisons la reconversion avant de sortir les résultats.

Les différentes propositions sont:
\begin{enumerate}
\item
Les examens: $x \in X$
\item
Les salles: $s \in S$
\item
Les périodes temporelles: $t \in \{0,...,T-1\}$
\end{enumerate}

La proposition qui les regroupe:
\begin{equation}
p = \{ p \underset{x,s,t}{} | x \in X, s \in S, t \in \{0,...,T-1\} \}
\end{equation} 


\subsubsection{Expression des différentes contraintes}
Les différentes contraintes sont reprises ensemble avec la concaténation de FNC, ce qui donne bien une FNC.
\begin{enumerate}
\item
Chaque examen a lieu une fois:
\begin{equation}
\bigwedge\limits_{x \in X}\ (\bigvee \limits_{s \in S}\ \bigvee \limits_{t \in \{0,...,T-1\} }\ ( p\underset{x,s,t}\ ))
\end{equation}
\item
Chaque examen a lieu une seule fois pour une seule salle et une seule période.
% Ici j'hésite à rajouter s1 \neq s2 et t1 \neq t2
% On a transformé \neg( a AND b) => (neg a OR neg b)
\begin{equation}
\bigwedge\limits_{x \in X}\ \bigwedge\limits_{s1,s2 \in S}\ \bigwedge\limits_{t1,t2 \in \{0,...,T-1\}}\ ( \neg p\underset{x,s1,t1}\ \lor \neg p\underset{x,s2,t2}\ )
\end{equation}
\item
Les examens ne peuvent pas se dérouler dans des pièces de capacité insuffisante.
Nous utilisons la fonction définie plus haut.
%Si on peut reprendre directement la fonction définit plus tôt:
\begin{equation}
\bigwedge\limits_{x \in X}\ \bigwedge\limits_{s \in S, l(x) > c(s)}\ \bigwedge\limits_{t \in \{0,...,T-1\}}\ ( \neg p\underset{x,s,t}\ )
\end{equation}
\item
Les examens d'un même étudiant ne peuvent pas avoir lieu en même temps.
% Je n'arrive pas à représenter que les exams sont du même étudiant
% même idée 2)
\begin{equation}
\bigwedge\limits_{e \in E}
\bigwedge\limits_{x1,x2 \in X, x1,x2 \in a(e), x1 \neq x2}\ \bigwedge\limits_{s1,s2 \in S}\ \bigwedge\limits_{t \in \{0,...,T-1\}}\ ( \neg p\underset{x1,s1,t}\ \lor \neg p\underset{x2,s2,t}\ )
\end{equation}
\item
Les examens d'un même professeur ne peuvent pas avoir lieu en même temps.
\begin{equation}
\bigwedge\limits_{e \in P}
\bigwedge\limits_{x1,x2 \in X, x1,x2 \in b(e), x1 \neq x2}\ \bigwedge\limits_{s1,s2 \in S}\ \bigwedge\limits_{t \in \{0,...,T-1\}}\ ( \neg p\underset{x1,s1,t}\ \lor \neg p\underset{x2,s2,t}\ )
\end{equation}
\item
Deux examens différents ne peuvent pas avoir la même salle en même temps.
\begin{equation}
\bigwedge\limits_{x1,x2 \in X, x1 \neq x2}\ \bigwedge\limits_{s \in S}\ \bigwedge\limits_{t \in \{0,...,T-1\}}\ ( \neg p\underset{x1,s,t}\ \lor \neg p\underset{x2,s,t}\ )
\end{equation}
\end{enumerate}

%chaque examen a lieu une fois :
%Pour tout examen, clause :
%[examen][0][0] V [examen][0][1] V ... V [examen][0][T] V [examen][1][0] V ... V [examen][1][T-1]
% V ... V [examen][0][S-1][0] V ... V [examen][0][S-1][T-1]
%Si on doit même éviter le "pour tout examen" :
%[0][0][0] V [0][0][1] V ... V [0][0][T] V [0][1][0] V ... V [0][1][T-1]
% V ... V [0][S-1][0] V ... V [0][S-1][T-1]
% ^ 
%[1][0][0] V [1][0][1] V ... V [1][0][T] V [1][1][0] V ... V [1][1][T-1]
% V ... V [1][S-1][0] V ... V [1][S-1][T-1]
% ^ ... ^
%[X-1][0][0] V [X-1][0][1] V...V [X-1][0][S-1][T-1]

%chaque examen a lieu une seule fois
% (NOT [0][0][0] V NOT [0][0][1]) ^ (NOT [0][0][0] V NOT [0][0][2]) ^ ... ^ (NOT [0][0][0] V NOT [0][0][T-1])
% ^ (NOT [0][0][0] V NOT [0][1][1]) ^ (NOT [0][0][0] V NOT [0][1][2]) ^ ... ^ (NOT [0][0][0] V NOT [0][1][T-1])
% ^ ...
% ^ (NOT [0][0][0] V NOT [0][S-1][1]) ^ (NOT [0][0][0] V NOT [0][S-1][2]) ^ ... ^ (NOT [0][0][0] V NOT [0][S-1][T-1])
% ^ (NOT [0][0][1] V NOT [0][0][0]) ^ (NOT [0][0][1] V NOT [0][0][2]) ^ ... ^ (NOT [0][0][0] V NOT [0][1][T-1])
% etc, trop marre, trop chaud si on n'utilise pas les "pour tout"
\subsection{Implémentation du problème}
Nous avons récupéré le code du parseur fournit, en ajoutant le parsing des intervalles interdits ainsi
que de la valeur k. La classe SchedSpec a été reprise dans un fichier à part en lui ajoutant des méthodes
pour ajouter les contraintes à un solveur et récupérer les solutions d'un solveur.
\lstinputlisting{../src/SchedSpec.hpp}
En fonction de la version de Bison utilisé, l'include de Parser.h dans Parser.l doit éventuellement être changé en Parser.hpp.
Un Makefile est disponible pour compiler notre code.


\subsubsection{Exemples}
\begin{lstlisting}[frame=single]
6; //T
2; //S
30;2; //c
2; //E
4; //P
4; //X
1,2,3; 2,3,4; //a
1;2;3;4; //b
\end{lstlisting}
Nous donneras $1,4;1,3;1,2;1,1;$.
\begin{lstlisting}[frame=single]
2; //T
2; //S
30;2; //c
2; //E
4; //P
4; //X
1,2,3; 2,3,4; //a
1;2;3;4; //b
\end{lstlisting}
Nous donneras $0$.

\subsection{Ajout d'une durée variable}
Cette implémentation est optimisable car il y a beaucoup de répétitions de clauses, 
mais nous laissons le solveur se charger des simplifications.
%tentative de reproduire ton développement
\begin{equation}
\bigwedge\limits_{x \in X}\ \bigwedge\limits_{s \in S}\ \bigwedge\limits_{t \in \{0,T-1\}}\ ( \bigwedge\limits_{d \in \{0, d(x)-1\}}\ (\neg p\underset{x,s,t}{} \bigvee\limits_{k \in \{d, d+d(x)-1\}, t-(d(x)-1)+k < T}\  (p\underset{x,s,t-(d(x)-1)+k}{})))
\end{equation}
%[examen][salle][temps] ==> ([examen][salle][temps-(duree-1)] ^ [examen][salle][temps-(durée-1)+1] ^ ... ^ [examen][salle][temps]) V ([examen][salle][temps-(duree-1)+1] ^ ... ^ [examen][salle][temps+1]) V ... V ([examen][salle][temps] ^ ... ^ [examen][salle][temps+(duree-1)]
%FNC : après distribution :
% (NOT [examen][salle][temps] V [examen][salle][temps-(duree-1)] V ... V [examen][salle][temps]) 
%soit pas examen a cette période là, soit aussi la première position de l'intervale de durée de l'examen
% ^ (NOT [examen][salle][temps] V [examen][salle][temps-(durée-1)+1] V ... V [examen][salle][temps+1])
%soit pas examen a cette période là, soit aussi un de ceux en deuxième position de l'intervale de durée de l'examen
% ^ ...
% ^ (NOT [examen][salle][temps] V [examen][salle][temps] V ... V [examen][salle][temps+(duree-1)])
%soit pas examen, soit au moins un de ceux en dernière position dans l'intervale

\subsubsection{Exemples}
\begin{lstlisting}[frame=single]
6; //T
2; //S
30;2; //c
2; //E
4; //P
4; //X
1,2,3; 2,3,4; //a
1;2;3;4; //b
1;1;2;1; //d
\end{lstlisting}
Nous donneras $2,6;1,4;1,2;1,1;$.
\begin{lstlisting}[frame=single]
6; //T
2; //S
30;2; //c
2; //E
4; //P
4; //X
1,2,3; 2,3,4; //a
1;2;3;4; //b
1;4;2;1; //d
\end{lstlisting}
Nous donneras $0$.


\subsection{Ajout d'intervalles de temps non-disponibles}
%Pour tout examen, pour toute salle, pour toute période appartenant à un intervalle interdit
%phi = phi ^ NOT [examen][salle][periode] (concaténation avec les clauses précédentes, à faire partout ?)
%pour moi il suffit de tout prendre mais indiquer dans {0,..,T-1} qu'on enlève {d^1\underset{i},d^2\underset{i}} (1 \le i \le n )
\begin{equation}
\bigwedge\limits_{x \in X}\ \bigwedge\limits_{s \in S}\ \bigwedge\limits_{t \in [d_i^1,d_i^2] (1 \le i \le n ) }\ (\neg p\underset{x,s,t}\ )
\end{equation}

\subsubsection{Exemples}
\begin{lstlisting}[frame=single]
6; //T
2; //S
30;2; //c
2; //E
4; //P
4; //X
1,2,3; 2,3,4; //a
1;2;3;4; //b
1;1;2;1; //d
1,1; //i
\end{lstlisting}
Nous donneras $2,6;1,5;1,3;1,2;$.
\begin{lstlisting}[frame=single]
6; //T
2; //S
30;2; //c
2; //E
4; //P
4; //X
1,2,3; 2,3,4; //a
1;2;3;4; //b
1;4;2;1; //d
1,3; //i
\end{lstlisting}
Nous donneras $0$.
\subsection{Ajout d'un nombre maximum de changements de locaux (ces pauvres élèves)}

Nous définissons tous les ensembles de k+2 variables.
Nous posons 

\begin{equation}
V = \{((x_1,s_1,t_1), ..., (x_{k+2},s_{k+2},t_{k+2}))\ |\ \forall i \in \{1,k+2\} : t_{i+1} > t_i, s_{i+1} \neq s_i, \forall j \in \{1,k+2\}, i \neq j : x_i \neq x_j
\end{equation}
Il s'agit donc de l'ensemble des tuples représentant k+1 changements de salles.

\begin{equation}
\bigwedge\limits_{v \in V}\ (\bigvee\limits_{(x_i,s_i,t_i) \in v}\  \neg p\underset{x_i,s_i,t_i}{})
\end{equation}
%Pour tout ensemble de k+2 variables (x,s,t), avec ti+1 > ti, si+1 != si, xi != xj 
%(on progresse dans le temps, sur toutes les salles différents de la précédentes, en ne prenant en compte que les examens non déjà pris)
%phi = phi ^ ( NOT (x0,s0,t0) V NOT (x1,s1,t1) V ... V NOT (xk+1,sk+1,tk+1)

\subsubsection{Exemples}
\begin{lstlisting}[frame=single]
6; //T
2; //S
30;2; //c
2; //E
4; //P
4; //X
1,2,3; 2,3,4; //a
1;2;3;4; //b
1;1;2;1; //d
1,1; //i
1; //k
\end{lstlisting}
Nous donneras $2,6;1,5;1,3;1,2;$.
\begin{lstlisting}[frame=single]
4; //T
2; //S
30;2; //c
2; //E
4; //P
4; //X
1,2,3; 2,3,4; //a
1;2;3;4; //b
1;4;2;1; //d
1,1; //i
1; //k
\end{lstlisting}
Nous donneras $0$.
\subsection{Utilisation du mode MAX-SAT}
Le mode MAX-SAT permet de distinguer les clauses dites hard des clauses dites soft.
Les clauses soft peuvent ne pas être satisfaites, mais le solveur cherchera tout de 
même à en satisfaire un maximum.

Nous n'avons pas pu maitriser les outils suffisament rapidement pour en montrer
une implémentation. Toutefois, le concept guidant l'implémentation aurait été le suivant :
toutes les clauses en dehors de la contrainte du nombre de changements de salle
auraient été définies hard. Pour minimiser le nombre de changement de salle,
il faut maximiser le nombre de transitions possibles sans changement de salle.
Ainsi, nous pouvons générer tous les couples [(x1,s1,t1), (x2,s2,t2)] tels que 
$x1 \neg x2$, $s1 \neg s2$, $t2 > t1$. En ajoutant, pour chacun de ces couples,
une clause disjonctive reprenant la négation de chacun de ses litéraux et marquée comme soft,
nous incitons le solveur à minimiser le nombre de changements de salle.

\subsection{Unité de temps entre les examens d'un même étudiant dans une salle différente}
Cette contrainte diffère peu de la contrainte sur la non-simultanéité des examens d'un même étudiant.
Il suffit d'étendre cette non-simultanéité à une période de temps supplémentaire.

\begin{equation}
\bigwedge\limits_{e \in E}
\bigwedge\limits_{x1,x2 \in X, x1,x2 \in a(e), x1 \neq x2}\ \bigwedge\limits_{s1,s2 \in S, s1 \neq s2}\ \bigwedge\limits_{t \in \{0,...,T-2\}} ( 
(\neg p\underset{x1,s1,t}\ \lor \neg p\underset{x2,s2,t}) \wedge 
(\neg p\underset{x1,s1,t}\ \lor \neg p\underset{x2,s2,t+1} ) )
\end{equation}

\subsubsection{Exemples}
\begin{lstlisting}[frame=single]
6; //T
2; //S
30;2; //c
2; //E
4; //P
4; //X
1,2,3; 2,3,4; //a
1;2;3;4; //b
1;1;2;1; //d
1,1; //i
1; //k
\end{lstlisting}
Nous donneras $1,6;1,5;1,3;1,2;$.
\begin{lstlisting}[frame=single]
5; //T
2; //S
30;2; //c
2; //E
4; //P
4; //X
1,2,3; 2,3,4; //a
1;2;3;4; //b
1;4;2;1; //d
1,1; //i
1; //k
\end{lstlisting}
Nous donneras $0$.
\subsection{Ajout d'une contrainte naturelle supplémentaire}
Certains enseignants plus agés ou habitants plus loin ne souhaitent pas être
présents sur le campus trop tôt ou trop tard.
Pour chaque enseignant, nous avons une fonction $q$ qui donne un ensemble de deux périodes pour lequel
ils ne seront pas présents. 
%modelisation logique : pour tout enseignant, pour tout examen de cet enseignant, pour toute salle, pour toute période dans l'ensemble, NOT mu(examen) = (salle, période)
Pour la modélisation logique:
\begin{equation}
\forall e \in P, \forall x \in X,  \forall s \in S,  \forall t \in \{q1,q2 \in q(e)\} | \neg ( \mu(x) = ( s, t )) \\
\end{equation}
Maintenant la formule sous forme FNC:
\begin{equation}
\bigwedge\limits_{e \in P}\ \bigwedge\limits_{x \in X, x \in b(e)}\ \bigwedge\limits_{s \in S}\ \bigwedge\limits_{t \in \{q1,q2 \in q(e)\}} ( \neg p\underset{x,s,t}{} )
\end{equation}
\subsection{Le problème est NP-dur}
%NP-dur : un NP-complet peut être traduit en ce pb
Le problème SAT général est un problème NP-complet d'après le théorème de Cook. Or, nous venons de prouver
que le problème d'emploi du temps peut se réduire à un problème SAT en un temps polynomial. Dans l'autre sens, tout problème SAT peut être traduit en problème
d'emploi du temps en utilisant autant de cases temporelles qu'il y a de variables booléennes. Toutes les contraintes SAT peuvent être traduites en contraintes sur
des périodes de temps.
Par conséquent, le problème d'emploi du temps est NP-dur.
\end{document}
